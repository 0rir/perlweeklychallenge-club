\documentclass{article}
%\usepackage[paperwidth=8.5in,paperheight=39in]{geometry}
\pagestyle{empty}
\setlength{\parindent}{0pt}
\setlength{\parskip}{1ex}
\begin{document}
\part*{The Counter to Equilibrium}
\section*{Task 2: Equal Distribution}
\textbf{Submitted by:} Mohammad S Anwar

You are given a list of integers greater than or equal to zero, $@list$.

Write a script to distribute the number so that each members are same.
If you succeed then print the total moves otherwise print $-1$.

Please follow the rules (as suggested by Niels 'PerlBoy' van Dijke) 
\begin{itemize}{}{}
\item You can only move a value of $1$ per move
\item You are only allowed to move a value of $1$ to a direct
  neighbor/adjacent cell 
\end{itemize}

\subsubsection*{Example 1}
Input: $@list = (1, 0, 5)$ \\
Output: $4$

Move \#1: $(1, 1, 4)$ \\
(2nd cell gets $1$ from the 3rd cell) \\
Move \#2: $(1, 2, 3)$ \\
(2nd cell gets $1$ from the 3rd cell) \\
Move \#3: $(2, 1, 3)$ \\
(1st cell gets $1$ from the 2nd cell) \\
Move \#4: $(2, 2, 2)$ \\
(2nd cell gets $1$ from the 3rd cell)

\subsubsection*{Example 2}
Input: $@list = (0, 2, 0)$\\
Output: $-1$

It is not possible to make each same.

\subsubsection*{Example 3}
Input: $@list = (0, 3, 0)$\\
Output: $2$

Move \#1: $(1, 2, 0)$ \\
(1st cell gets $1$ from the 2nd cell) \\
Move \#2: $(1, 1, 1)$ \\
(3rd cell gets $1$ from the 2nd cell)

\section*{Definitions}
A \textit{move}, i.e. the transfer of a unit from one element to one
of its immediate neighbors will be called \textit{shift left} or
\textit{shift right} in the following.

Given a distribution $(a_1,\ldots, a_n)$, we define the
\textit{cumulative distribution}
\begin{displaymath}
  (s_1,\ldots,s_n) := (a_1, a_1 + a_2,\ldots,a_1 +\ldots+ a_n) =
  \left(\sum_{i=1}^k a_i \right)_{k=1}^n
\end{displaymath}

\section*{Properties}
\subsection*{Moves}
Consider an element $a_i > 0$.

We can perform a \textit{shift left} from position $i$ to $i - 1$ if
$i > 1$: 
\begin{displaymath}
  a_k' = \left\{
    \begin{array}{ll}
      a_k + 1 & \textrm{if } k = i - 1 \\
      a_k - 1 & \textrm{if } k = i \\
      a_k & \textrm{otherwise}
    \end{array}
  \right.
\end{displaymath}
In terms of $s$:
\begin{displaymath}
  s_k' = \left\{
    \begin{array}{ll}
      s_k + 1 & \textrm{if } k = i - 1 \\
      s_k & \textrm{otherwise}
    \end{array}
  \right.
\end{displaymath}
We can perform a \textit{shift right} from positon $i$ to $i + 1$ if
$i < n$: 
\begin{displaymath}
  a_k' = \left\{
      \begin{array}{ll}
        a_k - 1 & \textrm{if } k = i \\
        a_k + 1 & \textrm{if } k = i + 1 \\
        a_k & \textrm{otherwise}
      \end{array}
    \right.
\end{displaymath}
In terms of $s$:
\begin{displaymath}
  s_k' = \left\{
    \begin{array}{ll}
      s_k - 1 & \textrm{if } k = i \\
      s_k & \textrm{otherwise}
    \end{array}
  \right.
\end{displaymath}

\subsection*{Effects}
Now we know exactly how a single move affects the cumulative
distribution: 
\begin{itemize}{}{}
\item A \textit{shift left} increments the cumulative distribution at
  the target position by one.
\item A \textit{shift right} decrements the cumulative distribution at
  the source position by one.
\item All other values of the cumulative distribution remain unchanged.
\end{itemize}

\section*{Disequilibrium}

There exists an equilibrium distribution if and only if
\begin{displaymath}
  s_n \bmod n = 0  
\end{displaymath}
Let's assume there is an equilibrium distribution for the given
numbers.
Then there is a \textit{cumulative equilibrium distribution}
\begin{displaymath}
  (e_1,\ldots,e_n) := (s_n / n, 2 s_n / n,\ldots,s_n)
  = \left(k s_n / n\right)_{k=1}^n
\end{displaymath}
Note that $e_n = s_n$ and $e_i < e_{i+1}$ for $i = 1,\ldots,n-1$.

Considering the cumulative distribution:
\begin{enumerate}
\item 
Suppose there is an $i$ with $s_i > e_i$.
Then $i < n$ because $s_n = e_n$.
If $a_i > 0$ then a \textit{shift right} from position $i$ to $i + 1$
decrements $s_i$ by one.
Otherwise (if $a_i = 0$), $s_{i - 1} = s_i > e_i > e_{i - 1}$ and we
may repeat the consideration at $i - 1$.
At least one of $a_1,\ldots,a_i$ must be non-zero because otherwise
$s_i > e_i$ cannot hold. 

\textbf{Decrementing:} \\
If there is an $i$ with $s_i > e_i$, then there is a $j$ with
$1 \leq j \leq i$, $a_j > 0$ and $s_i = s_j > e_j$.
With a \textit{shift right} from position $j$ to $j + 1$, $s_j$ can be
decremented by one.

\textbf{Example:}
\begin{eqnarray*}
  a & = & (2, 7, 0, 0, 1) \\
  s & = & (2, 9, 9, 9, 10) \\
  e & = & (2, 4, 6, 8, 10)
\end{eqnarray*}
Select $i = 4$: $s_4 = 9 > 8$ but $a_4 = 0$.
Looking left, we arrive at $j = 2$ having $s_2 = 9 > 4$ and $a_2 > 0$
where we can perform a \textit{shift right} from position $2$ to $3$
resulting in
\begin{eqnarray*}
  a' & = & (2, 6, 1, 0, 1) \\
  s' & = & (2, 8, 9, 9, 10)  
\end{eqnarray*}

\item
Suppose there is an $i$ with $s_i < e_i$.

Then $i < n$ because $s_n = e_n$.
If $a_{i + 1} > 0$ then a shift left from position $i + 1$ to $i$
increments $s_i$ by one. 
Otherwise (if $a_{i + 1} = 0$), $s_i = s_{i + 1} < e_i < e_{i + 1}$
and we may repeat the consideration at $i + 1$.
At least one of $a_{i + 1},\ldots,a_n$ must be non-zero because
otherwise $s_i < e_i$ cannot hold.

\textbf{Incrementing:} \\
If there is an $i$ with $s_i < e_i$, then there is a $j$ with
$i \leq j < n$, $a_{j + 1} > 0$ and $s_i = s_j < e_j$.
With a \textit{shift left} from position $j + 1$ to $j$, $s_j$ can be
incremented by one.

\textbf{Example:}
\begin{eqnarray*}
  a & = & (1, 0, 0, 7, 2) \\
  s & = & (1, 1, 1, 8, 10) \\
  e & = & (2, 4, 6, 8, 10)
\end{eqnarray*}
Select $i = 1$: $s_1 = 1 < 2$ but $a_2 = 0$
Looking right, we arrive at $j = 3$ having $s_3 = 1 < 6$ and $a_4 > 0$
where we can perform a \textit{shift left} from position $4$ to $3$
resulting in
\begin{eqnarray*}
  a' & = & (1, 0, 1, 6, 2) \\
  s' & = & (1, 1, 2, 8, 10)  
\end{eqnarray*}
\end{enumerate}

\subsection*{Existence}
From the previous section we already know that a single move
increments or decrements exactly one element of the cumulative
distribution by one. 

Now we have shown that in a distribution deviating from the
cumulative equilibrium distribution, there always exists a legal
move that reduces the overall deviation. 

\section*{Solution}
We have shown that for every deviation from the equilibrium
distribution there is a move that reduces the difference between the
cumulative distribution and the cumulative equilibrium distribution by
one.
The number of moves to achieve the equilibrium distribution is
therefore the sum of the absolute differences between the cumulative
distribution and the cumulative equilibrium distribution:
\begin{displaymath}
  m = \sum_{i=1}^n |s_i - e_i|
\end{displaymath}
This is a nice result for mathematicians: It gives the required number
of moves without actually providing them.
(However, from the above discussion it becomes clear that any legal
move reducing the overall deviation may be chosen at any time.)
\end{document}